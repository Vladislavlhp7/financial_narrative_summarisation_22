\section{Design \& Development}\label{sec:design-and-development}

\subsection{Data}\label{subsec:data}
The data for the FNS21 task is a collection of narrative parts of annual reports, converted from PDF to plain text.
As discussed in Section~\ref{subsec:fns} due to the rich visual representations in the PDFs, the resulting text suffers from various problems like
\begin{enumerate*}
    \item \emph{spacing inconsistencies} - mixing of tab-space word delimiters, over-segmentation (i.e., a split into incoherent chunks) and under-segmentation (i.e., merging of unrelated words),
    \item \emph{symbol encoding issues} - introduction of unreadable non-alphanumeric characters, and
    \item \emph{formatting issues} - words having different casing, hyphenation at the end of a line, etc
    \item \emph{conversion of tables to text} - financial figures spanning over multiple lines and being mixed with the text.
\end{enumerate*} (Figure~\ref{fig:pdf_to_text}).

% PDF-to-text conversion issues
\begin{figure}[ht]
    \centering
    \begin{enumerate}
        \item \begin{verbatim}
         Following my appointment as Chief
        E x e c u t i v e	in	J u l y	2 0 1 0 ,	g r e ate r	e m p h a s i s
        h a s	b e e n	p l a c e d	o n	f u l fi l l in g	t h e	s u p p l y	o f
        tonnage due under legacy contracts and
    \end{verbatim}
    \item \begin{verbatim}
        However,   the   Directors   further   believe   that   additional
        capital   	could   	be	  deployed	  to 	beneficial   	effect.
    \end{verbatim}
    \item \begin{verbatim}
        Opening net book amount 116,635 35,624 166,754 319,013
        Additions 51,380 7,647 307,546 366,573
    \end{verbatim}
    \item \begin{verbatim}
          This means that buyers can
        _0_@uk_ar06_front.indd   5 20/04/2007   09:13:30 05
        @UK PLC
        Annual Report and Accounts 2006
        use our network to purchase from their suppliers.
    \end{verbatim}
    \end{enumerate}
    \caption{PDF-to-text conversion issues.}
    \label{fig:pdf_to_text}
\end{figure}

To address these issues, we have developed a rigorous data cleaning pipeline that achieves the following key objectives:
\begin{enumerate*}
    \item handles space-tab mixing via hand-crafted rules (derived from observation\footnote{
      E.g., for some of the lines, characters were separated by spaces and words with tabs, hence the need for a custom rule.
    }),
    \item retains alphanumeric characters, punctuation, spaces, financial symbols, and
    \item removes sentences shorter than 3 words
\end{enumerate*}.

As discussed in Section~\ref{subsec:fns}, the annual reports are extremely long documents with an average length reported at 80 pages (\cite{litvak-vanetik-2021-summarization}).
Each one has at least two--three gold summaries provided by the FNS21 organisers, and we provide some statistics
\begin{enumerate*}
    \item helpful for grasping the nature of the output text, but also
    \item useful for the evaluation of the summarisation models
\end{enumerate*}.
On one hand, we can see that the average number of words in the longest summary is over $2,000$ (Figure~\ref{fig:longest_summary_word_count}), while the FNS21 regulations specify an expected output of at most $1,000$ words.
Furthermore, as we are not competing in the FNS21 task, for simplicity, during evaluation we will generate only summaries with at most 40 sentences.
We arrive at this number by observing that the median number of words in the longest summaries is 25 (Figure~\ref{fig:sentence_word_count}), and calculating that $\frac{1,000\text{words}}{25\text{words}}=40$ sentences can suffice.

\begin{figure}[ht]
    \begin{subfigure}{0.49\textwidth}
        \centering        \includegraphics[width=1\columnwidth]{../charts/longest_summary_word_count}
        \caption{Number of words in longest report summary}
        \label{fig:longest_summary_word_count}
    \end{subfigure}%
    \hfill
    \begin{subfigure}{0.49\textwidth}
        \centering
        \includegraphics[width=1\columnwidth]{../charts/sentence_word_count}
        \caption{Number of words in training sentences}
        \label{fig:sentence_word_count}
    \end{subfigure}
    \caption{Distribution of number of words in training sentences and report summaries}
    \label{fig:word_count}
\end{figure}

As we were only provided with the training and the validation FNS21 datasets (Table~\ref{tab:fns21-data}),
we decided to treat the validation set as a testing set (Table~\ref{tab:fns21-my-data})\footnote{
    We observed that two of the annual report files were empty, hence the difference of 3,361 and 3,363 (Table~\ref{tab:fns21-data} without testing set).
} and perform our own training-validation data split on a sentence level instead due to the significant variation in report lengths (\cite{litvak-vanetik-2021-summarization}).



\begin{table}[h]
    \centering
    \begin{tabular}{lrr r}
        \hline
        Data Type & Training + Validation & Testing & Total \\
        \midrule
        Report full text & 2,998 & 363 & 3,361 \\
        \bottomrule
    \end{tabular}
    \caption{Training-Validation-Testing Data Split}
    \label{tab:fns21-my-data}
\end{table}


\subsection{Methodology}\label{subsec:methodology}
We approach the annual report summarisation problem from a supervised perspective - we cast the task of Extractive Text Summarisation (ETS) as a binary classification problem defined on the sentence level.
More formally, we can describe the annual report as $d=\{s_{1}, s_{2}, \dots, s_{n}\}$, where $d$ is a document, represented in terms of sentences $s_{i}, \  1 \leq i \leq n$ (\cite{liu2019finetuningbert}).

Then, a candidate summary\footnote{
    A candidate summary is generated from a model $m_{i}$ but it is not yet a \emph{best summary}.
} can be $c=\{s_{1}, s_{2}, \dots, s_{k} | s_{i} \in d \}, \ 0 \leq k \leq n$.

We further need to define the \emph{gold summary}, $c^{*}$ for a document $d$.

In the case of the FNS21 task, there are at least two summaries per report, hence we will use the following notation for the set of all gold summaries for each document $C^{*} = \{c^{*}_{1}, c^{*}_{2}, \dots, c^{*}_{p}\}$.
Furthermore, the supervised learning labels are $y_{i} \in \{1,0\}$ for each sentence $s_{i}$ in $d$ if the sentence is or is not in \textbf{\emph{any}}\footnote{
    To increase the positive samples (i.e., the summarizing sentences) we do not restrict ourselves to just one gold summary in the training process unlike~\cite{orzhenovskii-2021-t5}.
    Our goal is to achieve better latent feature extraction of summaries through the employment of all existing data.
    However, we are aware that this approach is more likely to encounter standard ETS issues, specifically - extracted summary sentences could be retrieved from unrelated paragraphs in the report.
    This causes the \enquote{dangling anaphora} phenomenon, i.e. decontextualised extracts are stitched together and could mislead the reader due to out-of-context references as specified in~\cite{lin2009summarization}.
} of the gold summaries $c^{*}_{j}$ for that document.

In general, to assess the quality of a candidate summary $c$, we measure its similarity with the gold summary $c^{*}$ based on their n-gram overlap $R=(c, c^{*})$, where $R$ is the ROUGE-$F_{1}$\footnote{
    We use a slightly different but faster version of ROUGE compared to the official metric~\cite{lin2004rouge}.
    It can be accessed at: \url{https://github.com/pltrdy/rouge}
    } metric(\cite{lin2004rouge}).

For the FNS21 task due to the extractive nature of our approach we will evaluate our models based on the ROUGE-maximising $c^{*}_{i}$ gold summary\footnote{
    The intuition is that by extracting multiple sentences from the report, our generated candidate summary can retain sentences from \emph{any} of the gold summaries. 
    Hence, there must be at least one such gold summary where the overlap is maximal.
    The practical implications are that two models, $m_{1}$ and $m_{2}$ can produce two different candidate summaries $c_{1}$ and $c_{2}$, respectively.
    Their individual evaluation is based on gold summaries $c^{*}_{1}$ and $c^{*}_{2}$ (which can be the same when the candidates $c_{1}$ and $c_{2}$ are identical).
}, i.e.,

\begin{figure}[h]
    \centering
    \begin{equation}\label{eq:rouge_max}
        r = \underset{c^{*} \in C^{*}}{\operatorname{argmax}} R(c, c^{*}_{i})
    \end{equation}
    \caption{Candidate summary evaluation as a gold summary ROUGE-maximisation}
    \label{fig:rouge_max}
\end{figure}


While some authors (\cite{zmandar-etal-2021-joint}) follow the greedy ROUGE-maximisation method of matching summary sentences to document sentences (established in~\cite{nallapati2017summarunner}), we approach the problem in a more practical and faster fashion.
After manual observation of the reports against their gold summaries, it became clear that for almost all sentences of $c^{*}_{i}$, there was an exact match with a sentence in the whole annual report $d$.

This hypothesis was proven correct by one of the FNS21 contestants (\cite{orzhenovskii-2021-t5}) who reported that 99.4\% of the summaries were included in the
report as whole subsequences.
Hence, after having pre-processed the text documents we iteratively match the sentences and generate the binary classification labels ($\{1,0\}$ representing \emph{summary} and \emph{non-summary}, respectively) for both the training and testing datasets.
