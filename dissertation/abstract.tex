Although there has been considerable progress in Natural Language Processing (NLP) over the years, it has not fully reached the Accounting and Finance (AF) industry.
In the meantime, companies worldwide produce vast amounts of textual data as part of their reporting packages to comply with regulations and inform shareholders of their financial performance.
The glossy annual report is such an example, widely read by investors, but it also tends to be quite long.
Inspired by the Financial Narrative Processing (FNP) workshops (\cite{zmandar-etal-2021-financial, fnp-2022-financial}),
we design an Automatic Text Summarisation (ATS) system for the narrative parts of UK financial annual reports.
With this goal in mind, we implement and explore the following models for Extractive Text Summarisation (ETS):
\begin{enumerate*}
    \item attention-based Financial Recurrent Neural Network (FinRNN) with data augmentation,
    \item fine-tuned Financial BERT (FinBERT) (\cite{yang2020finbert})
\end{enumerate*}.
Our evaluations based on the ROUGE-2 metric show both models to be outperforming the standard ATS baselines: TextRank (\cite{mihalcea-tarau-2004-textrank}), and LexRank (\cite{Erkan2004LexRankGC}).
Furthermore, our FinBERT-base model achieves a ROUGE-2 score of $0.382$ beating \emph{the best English model in the FNS22} with $0.008$ (\cite{el-haj-etal-2022-financial}).
However, we must acknowledge that these results are not official because we were not provided with the FNS22 testing data,
and we had to create our own training-validation sets, while using the official validation set as our testing one.