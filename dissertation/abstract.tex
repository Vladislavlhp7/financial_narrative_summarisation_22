Although there has been considerable progress in Natural Language Processing (NLP) over the years, it has not fully reached the Accounting and Finance (AF) industry.
In the meantime, companies worldwide produce vast amounts of textual data as part of their reporting packages to comply with regulations and inform shareholders of their financial performance.
The glossy annual report is such an example, widely read by investors, but it also tends to be quite long.
Inspired by the Financial Narrative Summarisation 2021 (FNS21) Task (\cite{zmandar-etal-2021-financial}),
we design an Automatic Text Summarisation (ATS) system for the narrative parts of UK financial annual reports.
With this goal in mind, we implement and explore the following models for Extractive Text Summarisation (ETS):
\begin{enumerate*}
    \item attention-based Recurrent Neural Network (RNN) with data augmentation,
    \item fine-tuned Financial BERT (FinBERT) (\cite{yang2020finbert})
\end{enumerate*}.
Our evaluations based on the ROUGE-2 metric show both models to be outperforming the standard ATS baselines: TextRank (\cite{mihalcea-tarau-2004-textrank}), and LexRank (\cite{Erkan2004LexRankGC}).
Furthermore, our FinBERT model achieves a ROUGE-2 score of $0.382$ beating the FNS 2021 top-line (\cite{litvak-last-2013-multilingual}) with $0.10$, while also
being \emph{at least as good as the official best-performing model} in the FNS21 competition: the T5-LONG-EXTRACT (\cite{orzhenovskii-2021-t5}).
However, we must acknowledge that these results are not official because we were not provided with the FNS21 testing data,
and we had to create our own training-validation sets, while using the official validation set as our testing one.