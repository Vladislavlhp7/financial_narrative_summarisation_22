\section{Background}\label{sec:background}

\subsection{Supervised Learning}\label{subsec:supervised-learning}

\subsection{TFIDF}\label{subsec:tfidf}

\subsection{Word Embeddings}\label{subsec:word-embeddings}

\subsection{Attention}\label{subsec:attention}

\begin{figure}
    \begin{equation}
        \text{Attention}(Q, K, V) = \text{softmax}\left(\frac{QK^T}{\sqrt{d_k}}\right)V
    \end{equation}
    \caption{Attention calculation (Query, Key, Value)}
    \label{eq:attention}
\end{figure}

\subsection{Recurrent Neural Networks (RNNs)}\label{subsec:rnn}
The vanilla RNN is a basic type of RNN architecture designed for processing sequential data.
It learns temporal patterns from the initial data by looping over the hidden layers which allow information to persist (i.e., they serve as a network memory) (\cite{olah2015understandingLSTM}).
The key component is the recurrent hidden state updated at each time step using input data and the previous hidden state. 
This allows the RNN to capture contextual information and temporal dependencies in the sequence.
However, due to the inherent vanishing and exploding gradient problems with the vanilla RNNs, they have limited ability to learn long-term dependencies (\cite{bengio1994learning}).
To resolve these issues more advanced RNN architectures like LSTMs and GRUs have been developed. \\

The Long Short-Term Memory (LSTM) recurrent neural network has become a ubiquitous method in sequential problems (e.g., language modelling, time series forecasting).
This is so because it allows long-term dependencies to propagate through the network with the help of control gates \- \emph{input} and \emph{forget}, which reduce the effect of the vanishing gradient issue in the vanilla RNN\footnote{
    We direct readers to~\cite{bayer2015learning} where the authors demonstrate that the LSTM's \enquote{temporal} gradient is unaffected by the fixed weight factor $W$ of the vanilla RNN that is driving the derivative to zero. 
    This is ensured by the additional architecture unit \- the \emph{forget} gate, which learns to  control the gradient flow in the network.
}. 
A more simple variant of the LSTM is the Gated Recurrent Unit (GRU) which combines the \emph{input} and \emph{forget} gates into an \emph{update} gate for a model with fewer parameters and faster training (\cite{cahuantzi2021gru}).
Nevertheless, due to the sequential nature of the LSTM, the training process cannot be parallelised across GPUs, i.e., the learning cannot be made quicker by more computational resources.


\subsection{Transformers}\label{subsec:transformers}
The Transformer (\cite{vaswani2017attention}) is another sequence-to-sequence architecture which is parallelisable and attention-based. 



\begin{figure}
    \centering
    \begin{tikzpicture}[
  input/.style={rectangle, rounded corners, minimum height=2.5em, minimum width=3em, draw, fill=red!20},
  token/.style={rectangle, rounded corners, minimum height=2.5em, minimum width=3em, draw, fill=blue!20},
  seg_embed/.style={rectangle, rounded corners, minimum height=2.5em, minimum width=3em, draw, fill=orange!20},
  pos_embed/.style={rectangle, rounded corners, minimum height=2.5em, minimum width=3em, draw, fill=green!20},
  input_embed/.style={rectangle, rounded corners, minimum height=2.5em, minimum width=3em, draw, fill=purple!20},
  plus/.style={}
]

    % Tokens
    \node[input] (i1) {[CLS]};
    \node[input, right=0.5cm of i1] (i2) {It};
    \node[input, right=0.5cm of i2] (i3) {is};
    \node[input, right=0.5cm of i3] (i4) {sunny};
    \node[input, right=0.5cm of i4] (i5) {[SEP]};
    \node[input, right=0.5cm of i5] (i6) {But};
    \node[input, right=0.5cm of i6] (i7) {also};
    \node[input, right=0.5cm of i7] (i8) {cold};
    \node[input, right=0.5cm of i8] (i9) {[SEP]};

        % Tokens
    \node[token, below=1cm of i1] (t1) {$E_{[CLS]}$};
    \node[token, below=1cm of i2] (t2) {$E_{It}$};
    \node[token, below=1cm of i3] (t3) {$E_{is}$};
    \node[token, below=1cm of i4] (t4) {$E_{sunny}$};
    \node[token, below=1cm of i5] (t5) {$E_{[SEP]}$};
    \node[token, below=1cm of i6] (t6) {$E_{But}$};
    \node[token, below=1cm of i7] (t7) {$E_{also}$};
    \node[token, below=1cm of i8] (t8) {$E_{cold}$};
    \node[token, below=1cm of i9] (t9) {$E_{[SEP]}$};

    % Add plus nodes between inputs
    \draw ($(i1.south west) + (0,-0.3)$) -- ($(i9.south east) + (0,-0.3)$);

        % Segmentation embeddings
    \node[seg_embed, below=1cm of t1] (s1) {$E_{A}$};
    \node[seg_embed, below=1cm of t2] (s2) {$E_{A}$};
    \node[seg_embed, below=1cm of t3] (s3) {$E_{A}$};
    \node[seg_embed, below=1cm of t4] (s4) {$E_{A}$};
    \node[seg_embed, below=1cm of t5] (s5) {$E_{A}$};
    \node[seg_embed, below=1cm of t6] (s6) {$E_{B}$};
    \node[seg_embed, below=1cm of t7] (s7) {$E_{B}$};
    \node[seg_embed, below=1cm of t8] (s8) {$E_{B}$};
    \node[seg_embed, below=1cm of t9] (s9) {$E_{B}$};

    % Position embeddings
    \node[pos_embed, below=1cm of s1] (p1) {$E_{0}$};
    \node[pos_embed, below=1cm of s2] (p2) {$E_{1}$};
    \node[pos_embed, below=1cm of s3] (p3) {$E_{2}$};
    \node[pos_embed, below=1cm of s4] (p4) {$E_{3}$};
    \node[pos_embed, below=1cm of s5] (p5) {$E_{4}$};
    \node[pos_embed, below=1cm of s6] (p6) {$E_{5}$};
    \node[pos_embed, below=1cm of s7] (p7) {$E_{6}$};
    \node[pos_embed, below=1cm of s8] (p8) {$E_{7}$};
    \node[pos_embed, below=1cm of s9] (p9) {$E_{8}$};

    % Pluses at 0.25 between each (s, t) pair
    \node[above=0.25cm of $(s1)!.25!(t1)$, font=\large] {$+$};
    \node[above=0.25cm of $(s2)!.25!(t2)$, font=\large] {$+$};
    \node[above=0.25cm of $(s3)!.25!(t3)$, font=\large] {$+$};
    \node[above=0.25cm of $(s4)!.25!(t4)$, font=\large] {$+$};
    \node[above=0.25cm of $(s5)!.25!(t5)$, font=\large] {$+$};
    \node[above=0.25cm of $(s6)!.25!(t6)$, font=\large] {$+$};
    \node[above=0.25cm of $(s7)!.25!(t7)$, font=\large] {$+$};
    \node[above=0.25cm of $(s8)!.25!(t8)$, font=\large] {$+$};
    \node[above=0.25cm of $(s9)!.25!(t9)$, font=\large] {$+$};


    % Pluses at 0.25 between each (p, s) pair
    \node[above=0.25cm of $(p1)!.25!(s1)$, font=\large] {$+$};
    \node[above=0.25cm of $(p2)!.25!(s2)$, font=\large] {$+$};
    \node[above=0.25cm of $(p3)!.25!(s3)$, font=\large] {$+$};
    \node[above=0.25cm of $(p4)!.25!(s4)$, font=\large] {$+$};
    \node[above=0.25cm of $(p5)!.25!(s5)$, font=\large] {$+$};
    \node[above=0.25cm of $(p6)!.25!(s6)$, font=\large] {$+$};
    \node[above=0.25cm of $(p7)!.25!(s7)$, font=\large] {$+$};
    \node[above=0.25cm of $(p8)!.25!(s8)$, font=\large] {$+$};
    \node[above=0.25cm of $(p9)!.25!(s9)$, font=\large] {$+$};

\end{tikzpicture}
    \caption{BERT: Input Embeddings}
    \label{fig:bert_input}
\end{figure}

\subsection{Text Summarisation}\label{subsec:text-summarisation}
Text summarisation is the task of transforming a piece of text into a shorter  version that retains the most important information.
There are two overarching categories: extractive and abstractive text summarisation.
The former formulates the problem as a subset selection problem by returning only the most salient text excerpts from the original document (\cite{zhong-etal-2020-extractive}), while the latter aims to generate content anew, similar to how humans would do.

We will outline some key models that inspired our work below:
\begin{itemize}
    \item \textbf{Gokhan}: The authors employ an unsupervised summariser based on K-Means clustering of sentences encoded with SentenceBERT (\cite{reimers2019sentence}).
    However, their embeddings are pre-trained on general text, and they suggest that employing in-domain language models would result in a better performance.

    \item \textbf{AMUSE} (\cite{litvak-vanetik-2021-summarization}): The authors design an ETS system comprised of the following steps \begin{enumerate*}
        \item shortening of report with an existing Genetic Algorithm~\cite{litvak-last-2013-multilingual},
        \item encoding sentences with BERT vectors, and
        \item performing binary classification with LSTMs for salient sentence extraction
    \end{enumerate*}.
    They suggest that further work should incorporate \begin{enumerate*}
        \item efficient preliminary sentence removal, and
        \item additional neural modelling stages for the representation and detection of relevant input text parts.
    \end{enumerate*}

    \item \textbf{Hybrid model with RL} (\cite{zmandar-etal-2021-joint}): The authors train a joint extractive-abstractive summarisation model with reinforcement learning optimised for the ROUGE-2 F1 metric.
    Their networks are based on attentive LSTMs augmented with an additional copy mechanism (\cite{vinyals2015pointer}) achieving the second highest F1 score in the FNS21 competition.

    \item \textbf{T5} Hybrid (\cite{orzhenovskii-2021-t5}): The author used T5 (\cite{rayson2019t5}) for a hybrid model fine-tuned to generate the beginning of an abstractive summary and find the closest match of the output in the report’s full text.
    This is the best performing algorithm in the FNS21 competition but also the first to consider transformer models from an abstractive summarisation perspective in the FNP workshops so far.

\end{itemize}

In this work we will be solely exploring the extractive method, and more specifically - the \emph{supervised neural-based} (i.e., RNN, Transformer) type and the \emph{unsupervised graph-based} (i.e., TextRank, LexRank) type.


\subsection{LexRank}\label{subsec:lexrank}
LexRank (\cite{Erkan2004LexRankGC}) is an unsupervised extractive summarisation method consistently used as a baseline in the FNS21 and previous challenges.
It retrieves the most salient document sentences by computing their importance based on \emph{eigenvector centrality}.
To do that the algorithm creates a graph where each sentence represents a node and each edge is a weight between two nodes (\cite{Shearing2020AutomatedTS}).
The sentences are encoded as bag-of-words vectors of size $N$ - the vocabulary size, and the weight metric is a combination of tf-idf (Eq.\ref{eq:idf},\ref{eq:tfidf}) and cosine similarity - Eq.\ref{eq:cosinesimtfidf}.

\begin{equation}
    \text{idf}(t, D) = \log \frac{|D|}{|\{d \in D : t \in d\}|} \label{eq:idf}
\end{equation}

\begin{equation}
    \text{tf-idf}(t, d, D) = \text{tf}(t, d) \cdot \text{idf}(t, D)
    \label{eq:tfidf}
\end{equation}

\begin{equation}
    \text{tf\_idf\_cosine\_similarity}(s_1, s_2) = \frac{\sum_{t \in T} \text{tf-idf}(t, s_1, D) \cdot \text{tf-idf}(t, s_2, D)}{ \sqrt{\sum_{t \in T} \text{tf-idf}(t, s_1, D)^2} \cdot \sqrt{\sum_{t \in T} \text{tf-idf}(t, s_2, D)^2}}
    \label{eq:cosinesimtfidf}
\end{equation}

where $t$ is a term, $d$ is a document within a collection of documents/sentences $D$.

Also, $s_1$ and $s_2$ are two sentences and $T$ represents the set of all terms in both of them while $tf(t, d)$ denotes the term frequency of $t$ in $d$, and $idf(t, D)$ is the inverse document frequency of $t$ in the collection $D$.

The authors further propose finding the most important sentences by \begin{enumerate*}
    \item applying a threshold for the creation of edges with Eq.\ref{eq:cosinesimtfidf},
    \item building an adjacency matrix and normalizing it to produce \emph{transition probabilities},
    \item computing in an iterative fashion the \emph{eigenvector centrality} until convergence, and finally
    \item ranking sentences based on their \emph{lexical} PageRank (\cite{page1998anatomy}) score.
\end{enumerate*}