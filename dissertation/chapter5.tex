\chapter{Conclusion}\label{ch:conclusion}
In this chapter we summarise the main contributions of our work, highlighting its innovational aspects and discussing its limitations.

\section{Summary of Achievements}\label{sec:summary}
In this project we have explored the problem of summarising UK annual reports.
To deal with the significant amount of noise in the plain text of these glossy documents, we have built a rigorous pre-processing pipeline (Section~\ref{sec:data}).
We have further implemented a sentence extraction phase where we generate binary labels ($1$ being summary, $0$ - non-summary) from the reports and their multiple gold summaries (Section~\ref{sec:sentence_extraction}).
Once our datasets are created, we
\begin{enumerate*}[label=(\alph*)]
    \item design a Recurrent Neural Network (RNN) architecture (Section~\ref{sec:rnn_model}), and
    \item fine-tune a financial Transformer model - FinBERT (Section~\ref{sec:finbert})
\end{enumerate*},
training them in a supervised manner for a binary classification task (Sections~\ref{sec:training} and~\ref{sec:hyperparameters}).
We quantitatively evaluate our models with the ROUGE metric and demonstrate them outperforming traditional baselines (Chapter~\ref{ch:evaluation}).


Furthermore, at least for FinBERT we observe a clear ROUGE-2 improvement on the validation set over the best overall FNS22 model~\cite{foroutan-etal-2022-multilingual}.
Additionally, we show that our FinBERT also achieves competitive performance with the best FNS22 English model,
although noting that the our models are tested on different data (Section~\ref{sec:data}).
We also discuss the quality of the produced summaries (Section~\ref{sec:qualitative-discussion}), and in
Section~\ref{sec:limitations} we describe in more depth the limitations of our system and possible solutions.


In terms of \emph{innovational aspects} of our project in the context of the FNS challenge, we are the first to our knowledge that:
\begin{itemize}
    \item \emph{Integrate FinText word embeddings} - While some FNS21 competitors use general-domain sentence embeddings
    based on BERT~\cite{litvak-vanetik-2021-summarization, gokhan-etal-2021-extractive}, we represent sentences as
    a vector of word embeddings purpose-built for financial text analysis~\cite{rahimikia2021realised}.
    \item \emph{Perform back-translation as data augmentation} - In contrast to approaches where only the first 10\% of
    the annual report is used~\cite{orzhenovskii-2021-t5}, we over-sample the summarising sentences (minority class)
    by back-translating from French.
    \item~\emph{Fine-tune FinBERT~\cite{yang2020finbert} for extractive summarisation} - Transformer-based models have become increasingly
    popular in the FNP22 Workshop~\cite{khanna-etal-2022-transformer, pant-chopra-2022-multilingual},
    where some have used FinBERT for classifying definitions~\cite{ghosh-etal-2022-finrad},
    detecting hypernyms~\cite{peng-etal-2022-discovering}, and classifying financial sentiments~\cite{stepisnik-perdih-etal-2022-sentiment}.
    However, we are the first to adapt FinBERT for extractive summarisation.
\end{itemize}

\section{Discussion of Limitations}\label{sec:limitations}
Although, both our models have outperformed the baselines on ROUGE-2 and the fine-tuned FinBERT has achieved competitive
performance for the FNS22 task, there are several limitations that we would like to discuss:
\begin{itemize}
    \item \emph{Sentence embedding} - Although, we note our use of domain-specific FastText word embeddings due to their
    ability to handle noise and outperform general-domain embeddings~\cite{rahimikia2021realised}, we do not perform
    any sentence--level aggregation (i.e., dimensionality reduction) like averaging to condense the overall representation.
    While, this was a deliberate design decision to better capture the relationship between individual words,
    our sentence vectors became of size $(100, 300)$ instead of $(300,)$, which became more computationally expensive.
    Although, we are aware that FastText~\cite{bojanowski-etal-2017-enriching} provides average-pooling for any sequence,
    we were pessimistic of using it due to the loss of word order information (e.g., ``the company is good'' being represented just as ``is the company good'').
    Therefore, a limitation of our work is that we do not investigate the impact of using sentence embeddings
    (be it with positional encoding or average-pooling) on the performance of our models.
    \item \emph{Non-exhaustive evaluation} - Due to the FNS models being proprietary, and also evaluated on the official testing set,
    we are unable to make a more comprehensive comparison with the other models.
    However, in an ideal scenario we would perform further quantitative and qualitative evaluation of our performance.
    We would also like to investigate more in-depth the effect of under-sampling and the random generator on the classification capabilities of our models and
    compare with simply using the first $10\%$ of the annual reports as in~\cite{orzhenovskii-2021-t5}.
    \item \emph{Summary Generation} - In our work we take top $k$ sentences based on the model's output probability distribution (Chapter~\ref{ch:evaluation}).
        However, this is a very simplistic approach to summarisation that
        \begin{enumerate*}[label=(\alph*)]
            \item introduces incoherence issues (like the \emph{dangling anaphora phenomenon} from Section~\ref{sec:sentence_extraction}),
            \item trims the last sentence to fit the $1,000$ word limit, and it
            \item does not account for the \emph{informativeness} of the individual sentences
        \end{enumerate*}.
        To address the incoherence issues, we can try resolving the coreferences in the either through a
        graph-based approach on the generated summary~\cite{sonawane2016coreference}, or by introducing a more complex
        encoder architecture that represents and attends to entities as well as sentences~\cite{Huang2021ExtractiveSC}.
        Regarding the trimming heuristic, a natural improvement can be to use text compression techniques for the final predicted sentence~\cite{ghalandari2022efficient, KNIGHT200291}.
        As for the third point, we believe this is very exacerbating reason why our FinRNN architecture returns a summary
        without any single whole--sentence overlap with the gold standard (Fig.~\ref{fig:rnn_summary}).
        Instead, what~\cite{zmandar-etal-2021-joint} propose is a reinforcement learning approach which incorporates
        the \emph{sentence-level} ROUGE-2 score with the whole gold summary.
        While this method is much more sophisticated, it conveys the intuitive idea that the top-$k$ sentences comprising the
        \emph{optimal candidate summary} should be \emph{greedily maximising the global summary-level ROUGE-2} score.
\end{itemize}

\section{Future Work}\label{sec:future-work}
While in our project we only consider the narrative summarisation of
financial reports already converted to plain text, we propose the following pipeline as a direction for future work:
\begin{enumerate}
    \item \emph{PDF-to-Text} - Integrate into the summarisation system a PDF-to-Text conversion tool for annual
    reports like the CFIE-FRSE\footnote{\url{https://github.com/drelhaj/CFIE-FRSE}}~\cite{elhaj2019multilingual}, which
    also extracts the text into 8 generic section headers (Section~\ref{sec:uk-annual-reports}).
    \item \emph{Text-to-Summary} - Implement an extractive method that addresses the limitations from Section~\ref{sec:limitations},
    or alternatively, an abstractive method producing \emph{lay summaries} for non-expert users~\cite{vinzelberg2023lay, Guo2020AutomatedLL}.
    \item \emph{Text-to-Analysis} - Apply NLP techniques like sentiment analysis~\cite{araci2019finbert}, named entity recognition
   ~\cite{zhang2022finbertmrc}, and detection of forward-looking sentences~\cite{stihec-etal-2021-preliminary},
    to extract useful information from the summary and the text.
    Additionally, important financial disclosure characteristics as amount, tone, and transparency~\cite{li2010textual, Li2011TextualAO}
    would be beneficial for AF researchers and users.
    \item \emph{Packaged Software} - Build a drag-and-drop software application that allows users to upload a PDF file of an
    annual report, where the backend will perform the steps above and return a summary
    (see example Figures~\ref{fig:future-work-mockflow-report}, and~\ref{fig:future-work-mockflow-news} in Appendix).
    The suggested textual analysis features could be integrated as interactive visual elements.
    Furthermore, through recognising company names, the system could also provide dashboard of news and stock prices
    with the help of company-to-identifier mapping~\cite{el-haj2019retrieving}
    (i.e., getting the company ticker, e.g., \emph{NASDAQ: AAPL} for Apple Inc).
\end{enumerate}