\section{Introduction}\label{sec:introduction}

\subsection{Financial Reports} \label{subsec:financial_reports}
Due to international regulations, companies are obliged to report their periodic performance (annual, bi-annual, quarterly) to various regulatory authorities\footnote{Regulation authorities worldwide:
    \begin{itemize}
        \item Securities and Exchange Commission (SEC) in the USA
        \item European Securities and Markets Authority (ESMA) in Europe
        \item Financial Reporting Council (FRC) in the UK
        \item International Financial Reporting Standards (IFRS) in 167 jurisdictions worldwide
    \end{itemize}
    } and other users (e.g., corporate stakeholders, investors, customers, suppliers, etc.).
    These reports contain essential information about the operations and finances of a business and are crucial for making informed decisions (from a user perspective), but are different in regulatory forms.
    For example,
\begin{enumerate}
    \item 10-K reports filed to the SEC\footnote{\url{https://www.sec.gov}} and accessible through their Electronic Data Gathering, Analysis, and Retrieval\footnote{\url{https://www.sec.gov/edgar}} (EDGAR) system are only for US registered businesses.
    They follow a standardised template and are plain text, which makes them particularly easy for automated large-scale research (\cite{el-haj2019retrieving}).
    Also, the contents of these reports are strict, requiring solely five information sections\footnote{
    \begin{enumerate*}
        \item Business Overview
        \item Risk Factors
        \item Management's Discussion and Analysis of Financial Condition and Results of Operations (MD\&A)
        \item Financial Statements
        \item Supplementary Disclosures
    \end{enumerate*}
    }.
    \item UK annual reports, regulated by the UK's Financial Reporting Council (FRC), are typically the primary annual reporting method (also provided as PDF files).
    Unlike the 10-K, they are glossy and more stakeholder-oriented and enjoy unlimited discretion over non-mandated content (\cite{el-haj2019retrieving}) (e.g., photography and company brand material, non-mandatory narrative sections, etc.).
    However, these are more challenging for automated processing due to their variable section structure, formatting, and rich visual representations.
\end{enumerate}

\subsection{UK annual reports}\label{subsec:uk-annual-reports}
The annual report is the primary corporate disclosure legally required for public companies by regulatory authorities.
While it \emph{does not have a rigid document structure} like the 10-K, it typically has a \emph{narrative component}\footnote{The narrative component of a UK annual report typically consists of
\begin{enumerate*}
    \item Management's Commentary
    \item Letter to Shareholders
    \item Corporate Governance Statement
    \item Auditor's Report
    \item Remuneration Report
    \item Business Review
    \item Environmental, Social, and Governance (ESG) Report
    \item Risk Management Report
\end{enumerate*}
} and the financial statements (at the rear).

As we outlined in Section~\ref{subsec:financial_reports}, UK annual reports have the following inconvenient properties with regard to large-scale text understanding.
\begin{itemize}
    \item They are very long documents.
    Throughout the years, their average length has been increasing significantly with the number of pages rising 57\% for the median report from 2003 to 2016 (47 to 74 pages, respectively) (\cite{lewis_young_2019}), due to additional regulations between 2006 and 2008 (\cite{el-haj2019retrieving}.
    \item They have variable nomenclature.
    From firm to firm, naming conventions vary \enquote{dramatically}, with more than 20 unique titles for various sections (e.g., Chair's letter to shareholders, Management Commentary) (\cite{lewis_young_2019}).
    \item They incorporate embedded info-graphics.
    While domain experts hail the integration of highly interactive elements into corporate reporting (\cite{kriz2016future}), the compilation to PDF makes the task of analysing such unstructured documents automatically even harder (\cite{lewis_young_2019}).
\end{itemize}

These challenges motivate the work of~\cite{elhaj2019multilingual} who \begin{enumerate*}[label=(\alph*)]
    \item established a set of 8 generic section headers\footnote{
        \begin{enumerate*}
            \item Chairman Statement
            \item CEO Review
            \item Corporate Governance Report
            \item Directors Remuneration Report
            \item Business Review
            \item Financial Review
            \item Operating Review
            \item Highlights
        \end{enumerate*}
    } and
    \item built the CFIE-FRSE\footnote{
        The CFIE-FRSE stands for Corporate Financial Information Environment - Final Report Structure Extractor.
        It is publicly available at \url{https://github.com/drelhaj/CFIE-FRSE} and it can be used to convert English, Spanish and Portuguese annual reports.
    } extraction tool that converts a text-based PDF annual report to simple text.
\end{enumerate*}

\subsection{NLP in Accounting and Finance}\label{subsec:nlp-in-accounting-and-finance}
The relevance of this project should also be understood from the perspective of the development of Natural Language Processing (NLP) in the Accounting and Finance (AF) domain.
As outlined in~\cite{elliott1998accounting}, investors’ trust in the accountability of businesses would be based no longer as much on just the financial statements, but also on more descriptive narratives that define strategy and planning of resource use.
While some recognise the importance of understanding in-domain textual information (\cite{li2010textual}), others like~\cite{el-haj2019meaning} report that the industry is still doubtful and cynical about the NLP applications in the analysis of financial market disclosures.
Furthermore, the latter also observe that AF researchers rely extensively on bag-of-words models, which are \emph{not sufficient to encode complex contextual and semantic meaning} (especially in a domain with such \emph{specialized language}).
As for ATS~\cite{hollander-white-af} is said to be the single AF study into disclosure summarisation.
It demonstrates that machine-generated summaries are less likely to bias positively investor decisions compared to managerial ones.
Therefore, this only confirms the existence of a wide gap in NLP applications in Accounting research, which further motivates our work.

\subsection{Financial Narrative Summarisation 2021 (FNS21) Task} \label{subsec:fns}
The FNS Task is part of the annual Financial Narrative Processing (FNP) Workshop~\footnote{\url{https://wp.lancs.ac.uk/cfie/}} organised by Lancaster University since 2018, which aims to:
\begin{itemize}
    \item encourage the advancement of financial text mining \& narrative processing
    \item examine methods of structured content retrieval from financial reports
    \item explore causes and consequences of corporate disclosure
\end{itemize} as stated in their inaugural proceedings~\footnote{\url{https://wp.lancs.ac.uk/cfie/fnp2018/}}.

For that purpose, they produce datasets of extracted narratives (with the help of the CFIE-FRSE tool) from annual reports of UK companies listed on the London Stock Exchange (LSE).

In their FNS21 Task, there were $3,863$ such reports (Table~\ref{tab:fns21-data}), while the average length was reported at 80 pages, and the maximum of more than 250 pages (\cite{litvak-vanetik-2021-summarization}).

Additionally, for every report, there were at least two gold summaries situated in the annual report itself~\footnote{
    The gold summaries being already in the annual report is not problematic because these reports are already written by domain experts who know how to summarise the financial state of a company.
    Hence, multiple sections/paragraphs could achieve this thoroughly, and the organisers have identified \& extracted them manually with the help of the professional writers of the individual reports.
    At this moment, one can begin to doubt the point of applying ATS techniques, but due to the \emph{lack of rigid document structure}, \emph{it is not trivial to automatically find these text excerpts with heuristic methods}.
    Furthermore, we can formulate this challenge as finding the latent features of a summarising (i.e., \enquote{to-be-in-the-summary}) sentence, highlighted as one of the fundamental advantages of NLP in AF research (\cite{lewis_young_2019}, \cite{el-haj2019meaning}).
}
The workshop's goal was to build ATS systems that generate a single summary for an annual report, no longer than $1,000$ words (almost just as long as the gold summaries on average).

\begin{table}[h]
    \centering
    \begin{tabular}{lrrr r}
        \hline
        Data Type & Training & Validation & Testing & Total \\
        \midrule
        Report full text & 3,000 & 363 & 500 & 3,863 \\
        Gold summaries & 9,873 & 1,250 & 1,673 & 12,796 \\
        \bottomrule
    \end{tabular}
    \caption{FNS21 Data Split}
    \label{tab:fns21-data}
\end{table}

We acknowledge that due to the scarcity of publicly available financial data this third-year project could not have been possible without the kind permission of the FNP organisers to use the training and validation datasets from their FNS21 Task (\cite{fnp-2021-financial}).