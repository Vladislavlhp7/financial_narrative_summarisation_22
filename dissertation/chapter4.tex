\chapter{Evaluation}\label{ch:evaluation}
In this chapter we outline the evaluation process of our models.
For that purpose, we describe how we use the ROUGE metrics (Section~\ref{subsec:rouge}) to compute the
similarity between the gold and candidate summaries, and how we arrive at the final evaluation aggregation (Section~\ref{sec:evaluation-mechanism}).
Afterwards, we provide the results of our evaluation (Section~\ref{sec:quantitative-evaluation}), where we compare our models to the baselines and official FNS models.
Finally, we discuss the qualitative limitations of our produced summaries (Section~\ref{sec:qualitative-discussion}) before we conclude our work in Chapter~\ref{ch:conclusion}.

\section{Evaluation Mechanism}\label{sec:evaluation-mechanism}
Now we will provide a formal formulation of the summary generation process (Section~\ref{sec:summary-generation}) -
to assemble a candidate summary $c_{i}$ we prepare the sentences $s_{j}^{i}$ from a report $d_{m}$ (e.g., embed or tokenize sentences for the FinRNN and FinBERT, respectively).
Afterwards, we feed them into our model computing the output summary probabilities $p_{j}^{i}$, and then we select the
top-$k$ sentences ($k=40$, Section~\ref{sec:data}) based on the highest sentence probabilities $p_{j}^{i}$.
Finally, we concatenate them to form the summary $c_{i}$ in natural order (i.e., the order in the input text),
but also trim it to the maximum length of $1,000$ words (Section~\ref{sec:fns}).

In general, to assess the quality of a candidate summary $c$, we measure its similarity with the gold summary $c^{*}$
based on their subsequence (i.e., n-gram) overlap $R=(c, c^{*})$, where $R$ is the ROUGE-2\footnote{
        We use a slightly different but faster version of ROUGE compared to the official metric~\cite{lin2004rouge}.
        It can be accessed at: \url{https://github.com/pltrdy/rouge}. \\
        The FNS evaluation metric is the $F1$-score of ROUGE-2, and we will use it for the final evaluation.
} metric~\cite{lin2004rouge}.
For the FNS task due to the extractive nature of our approach we will evaluate our models based on
the ROUGE-L-maximising\footnote{
    We say ROUGE-L-maximising for conciseness, because ROUGE is a metric with precision and recall components, which are combined into a single $F1$ score.
    It is precisely the $F1$ score that we maximise.
    However, we are aware this introduces more complexity to our naming convention, hence we will use the term ROUGE-L-maximising.
} $c^{*\max}_{i}$ gold summary (Section~\ref{subsec:rouge}), i.e.,

\begin{equation}\label{eq:rouge_max}
    c^{*\max} = \underset{c^{*} \in C^{*}}{\operatorname{argmax}} \text{ROUGE-L}(c, c^{*}_{i})
\end{equation}
where $C^{*}$ is the set of gold summaries for a given report $d$, and $c^{*\max}_{i}$ is the gold summary with maximal ROUGE-L score with the candidate summary $c$.

The intuition is that by extracting multiple sentences from the report, our generated candidate summary can
retain sentences from \emph{any} of the gold summaries.
Hence, there must be at least one such gold summary where the longest common subsequence overlap (ROUGE-L) is maximal.
The practical implications are that two models, $m_{1}$ and $m_{2}$ can produce two different candidate summaries
$c_{1}$ and $c_{2}$, respectively.
Their individual evaluation is based on gold summaries $c^{*}_{1}$ and $c^{*}_{2}$ (which can be the same when the
candidates $c_{1}$ and $c_{2}$ are identical).
This guarantees that we are always comparing candidate summaries based on their maximal evaluation scores (i.e., their maximal summarising potential).

Therefore, the final evaluation score for a model $m$ is the average ROUGE-2 score\footnote{
    Here, once again we mean the $F1$-measure of ROUGE-2.
} between the candidate summaries $c_{i}$ and their corresponding \emph{ROUGE-L-maximising gold summaries} $c^{*\max}_{i}$, i.e.,
\begin{equation}\label{eq:rouge_final}
    r_{m} = \frac{1}{|C|} \sum_{i=1}^{|C|} \text{ROUGE-2}(c_{i}, c^{*\max}_{i})
\end{equation}
where $|C|$ is the number of candidate summaries $c_{i}$.

\section{Quantitative Evaluation}\label{sec:quantitative-evaluation}
Following this evaluation mechanism, we compare our models in terms of their ROUGE metrics against the baselines and official FNS models.
For that purpose, we produce two tables, namely:
\begin{itemize}
    \item Table~\ref{tab:rouge_performance_validation}, which compares \textbf{all} of our models against the baselines;
    \item Table~\ref{tab:rouge_performance_validation_fns}, which compares our \textbf{best} models against the FNS ones.
    Results are on the official validation set (used as a testing set for our models);
\end{itemize}
We again wish to remind the reader that we were not provided with the FNS22 testing set (Section~\ref{sec:data}).

In our FNS comparison we include the following models, namely the:
\begin{enumerate}
    \item T5 model~\cite{el-haj-etal-2022-financial} (testing data is only available), which is the completely based on the T5-LONG-EXTRACT and also the best English model in that edition;
    \item mT5 model~\cite{foroutan-etal-2022-multilingual}, which is the best model overall for all languages in the competition (multilingual for English, Spanish, and Greek);
    \item Longformer-Encoder-Decoder (LED)~\cite{khanna-etal-2022-transformer}
    \item Top-K Narrative Extractor (NE) ~\cite{shukla-etal-2022-dimsum}  ranking in the top three models overall for Spanish and Greek;
\end{enumerate}
We have the official English validation results for all but the T5~\cite{el-haj-etal-2022-financial}, for which we are going to use their testing evaluations (we are aware that this comparison is somewhat unfair).

\begin{table}[ht]
    \centering
    \begin{tabular}{lccc}
        \toprule
        \textbf{Model} & \textbf{ROUGE-1} & \textbf{ROUGE-2} & \textbf{ROUGE-L} \\
        \midrule
            TextRank ~\cite{mihalcea-tarau-2004-textrank} & 0.220 & 0.064 & 0.196 \\
            LexRank ~\cite{Erkan2004LexRankGC} & 0.250 & 0.086 & 0.227 \\
        \midrule
            FinRNN-base + attention & 0.221 & 0.062 & 0.204 \\
            FinRNN-base & 0.220 & 0.063 & 0.201 \\
            FinRNN-base + back-translation & 0.266 & 0.100 & 0.247 \\
            FinRNN-base + attention + back-translation & 0.276 & 0.106 & 0.249 \\
            FinBERT-base + back-translation & 0.490 & 0.321 & 0.468 \\
            FinBERT-base & \textbf{0.544} & \textbf{0.382} & \textbf{0.524} \\
        \bottomrule
    \end{tabular}\caption{ROUGE $F1$ Scores of our FinRNN and FinBERT models against baselines TextRank and LexRank}
    \label{tab:rouge_performance_validation}
\end{table}

\begin{table}[ht]
    \centering
    \begin{tabular}{lccc}
        \toprule
        \textbf{Model} & \textbf{ROUGE-1} & \textbf{ROUGE-2} & \textbf{ROUGE-L} \\
        \midrule
            TextRank ~\cite{mihalcea-tarau-2004-textrank} & 0.220 & 0.064 & 0.196 \\
            LexRank ~\cite{Erkan2004LexRankGC} & 0.250 & 0.086 & 0.227 \\
        \midrule
            mT5 ~\cite{foroutan-etal-2022-multilingual} & 0.440 & 0.301 & 0.423 \\
            LED ~\cite{khanna-etal-2022-transformer} & 0.442 & 0.302 & 0.434 \\
            T5 ~\cite{el-haj-etal-2022-financial} (\emph{official testing set}) & 0.496 & 0.374 & 0.487 \\
            Top-K NE ~\cite{shukla-etal-2022-dimsum} & \textbf{0.546} & \textbf{0.425} & - \\
        \midrule
            FinRNN-base + attention + back-translation (ours) & \emph{0.276} & \emph{0.106} & \emph{0.249} \\
            FinBERT-base (ours) & \emph{0.544} & \emph{0.382} & \emph{0.524} \\
        \bottomrule
    \end{tabular}\caption{ROUGE $F1$ Scores on official FNS22 validation set (used as testing in our models)}
    \label{tab:rouge_performance_validation_fns}
\end{table}

We can provide the following commentary on the results:
\begin{itemize}
    \item Our best performing model, FinBERT-base, which is a pre-trained on financial communication documents~\cite{yang2020finbert},
        achieves an average ROUGE-2 score of $0.382$, which outperforms by $0.081$ on the validation set the best performing
        model \textbf{overall} in the FNS22 competition - \emph{mT5}~\cite{foroutan-etal-2022-multilingual}.
        Furthermore, our FinBERT-base also seems to \emph{slightly outperform} the \textbf{best English model} in the FNS22 - T5~\cite{el-haj-etal-2022-financial} with $0.008$,
        though this comparison is not entirely fair as we are not using the same datasets.
        Regarding the other FNS models, we observe that our FinBERT-base beats the LED~\cite{khanna-etal-2022-transformer} with a similar margin as the mT5~\cite{foroutan-etal-2022-multilingual}.
        However, the Top-K NE ~\cite{shukla-etal-2022-dimsum} remains with the highest ROUGE-1 and ROUGE-2 $F1$ measures.
        Surprisingly, data augmentation does not improve the performance of our FinBERT-base (Table~\ref{tab:rouge_performance_validation}), and we believe this was caused by
        the \hyperlink{data_augment_hypothesis}{back-translation hypothesis} we made in Section~\ref{sec:hyperparameters}.
        Nevertheless, both models clearly outperform the official baselines: LexRank~\cite{Erkan2004LexRankGC} and TextRank~\cite{mihalcea-tarau-2004-textrank} (Table~\ref{tab:rouge_performance_validation}).
    \item The FinRNN-base + attention + back-translation model is the best performing model out of all recurrent neural architectures.
    While preliminary binary classification results did not show any considerable differences between the models, clearly
    \begin{enumerate*}[label=(\alph*)]
        \item the attention mechanism helps the model to better recognise the summarising sentences (i.e., attends to the most descriptive linguistic features), and
        \item the back-translation data augmentation significantly improves the practical performance of the model (i.e., the probability distribution of the summarising sentences),
              which is clearly not the case for our FinBERT model.
    \end{enumerate*}
    Additionally, we must note that except of the Top-K NE~\cite{shukla-etal-2022-dimsum}, all other FNS models are transformer-based,
    hence they have more complex architectures and attention mechanisms than our FinRNN-based models with its single-head attention (Sections~\ref{sec:transformers} and~\ref{sec:rnn_model}).
    \item At the same time, we acknowledge that the universal summarisation baselines: LexRank and TextRank, outperform
        our simple FinRNN models (Table~\ref{tab:rouge_performance_validation}), and we attribute this to both:
    \begin{enumerate}%[label=(\alph*)]
        \item the lack of sufficient descriptive training data from the positive class (i.e., the summarising sentences, Table~\ref{tab:random_under_sampling});
        \item the 90\% random under-sampling of the majority class data (see Section~\ref{sec:data});
        \item the bias towards long and complex sentences (Section~\ref{sec:qualitative-discussion});
        \item the summary generation process of selecting the top-$k$ sentences based on their sorted probabilities (Sections~\ref{sec:qualitative-discussion} and~\ref{sec:limitations});
    \end{enumerate}
\end{itemize}

%\newpage

\section{Qualitative Discussion}\label{sec:qualitative-discussion}
After having established the quantitative performance of our models, we now turn to the qualitative discussion of the results.
For a random annual report we generated a summary using the FinBERT-base + data augmentation model and
the FinRNN-base + attention model (see Figures~\ref{fig:finbert_summary},~\ref{fig:rnn_summary}).
where green colour indicates the summarising sentences, and red colour -- the non-summarising sentences).
We chose these models over the best ones because they have slightly lower ROUGE scores, and make more mistakes, which will help us identify the issues in the summarisation process.
We can make the following conclusions based on observation:
\begin{enumerate}
    \item The FinBERT model has \emph{around 50\% of its contents} belonging to the gold summary (Figure~\ref{fig:finbert_summary}),
        while all other sentences look very convincing in terms of their summarising potential (i.e., they are informative of the financial situation but also concise)
    \item At the same time, the FinRNN has the opposite characteristics:
    \begin{enumerate*}[label=(\alph*)]
        \item none of its sentences are in the gold summary (Figure~\ref{fig:rnn_summary}), while
        \item all of them are very long, containing uninformative but diverse sets of words, which in turn results in higher ROUGE scores
    \end{enumerate*}.
    This clearly represents the problem of using ROUGE as a metric for summarisation since it is only measures the lexical overlap, while being semantically unaware~\cite{akter-etal-2022-revisiting}.
\end{enumerate}
While we acknowledge that the FinRNN model seems to be \emph{biased towards long and noisy sentences}, we must note that in the example
only $5$ sentences have been generated to fit the word limit.
Therefore, we believe the summary generation process (i.e., the mechanism to combine predicted sentences into a single summary) further
exacerbates the accuracy of the summarisation.

Although, the practical results from Figures~\ref{fig:rnn_summary} might seem disappointing, we must once again remind the reader that the reports are extremely
long with an average number of sentences and words at around $2,700$ and $58,000$, respectively per annual report~\cite{litvak-vanetik-2021-summarization}.
Meanwhile, we are constrained to producing a summary of at most $40$ sentences (Section~\ref{sec:data}) or $1,000$ words (Section~\ref{sec:fns}).

\newpage