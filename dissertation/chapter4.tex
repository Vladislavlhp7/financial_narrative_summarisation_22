\section{Evaluation}\label{sec:evaluation}

In general, to assess the quality of a candidate summary $c$, we measure its similarity with the gold summary $c^{*}$
based on their n-gram overlap $R=(c, c^{*})$, where $R$ is the ROUGE-$F_{1}$\footnote{
    We use a slightly different but faster version of ROUGE compared to the official metric~\cite{lin2004rouge}.
    It can be accessed at: \url{https://github.com/pltrdy/rouge}
} metric(\cite{lin2004rouge}).

For the FNS21 task due to the extractive nature of our approach we will evaluate our models based on
the ROUGE--maximising $c^{*}_{i}$ gold summary, i.e.,

\begin{figure}[h]
    \centering
    \begin{equation}\label{eq:rouge_max}
        r = \underset{c^{*} \in C^{*}}{\operatorname{argmax}} R(c, c^{*}_{i})
    \end{equation}
    \caption{Candidate summary evaluation as a gold summary ROUGE-maximisation}
    \label{fig:rouge_max}
\end{figure}

The intuition is that by extracting multiple sentences from the report, our generated candidate summary can
retain sentences from \emph{any} of the gold summaries.
Hence, there must be at least one such gold summary where the overlap is maximal.
The practical implications are that two models, $m_{1}$ and $m_{2}$ can produce two different candidate summaries
$c_{1}$ and $c_{2}$, respectively.
Their individual evaluation is based on gold summaries $c^{*}_{1}$ and $c^{*}_{2}$ (which can be the same when the
candidates $c_{1}$ and $c_{2}$ are identical).

