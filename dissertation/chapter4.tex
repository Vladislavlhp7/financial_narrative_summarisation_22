\section{Evaluation}\label{sec:evaluation}

\section{Evaluation}\label{sec:evaluation2}

\subsection{Confusion Matrix}\label{subsec:confusion-matrix}
The confusion matrix is an essential tool to visualise and help assessing the performance of trained classifiers against the true labels $y_{i}$.

For the problem of binary classification, it is a square matrix (Table~\ref{tab:confusion_matrix}) that displays the following key elements:

\begin{itemize}
    \item True Positives (TP): Correct predictions of the positive class.
    \item True Negatives (TN): Correct predictions of the negative class.
    \item False Positives (FP): Incorrect predictions of the positive class (Type I error).
    \item False Negatives (FN): Incorrect predictions of the negative class (Type II error).
\end{itemize}

where for the problem of extractive text summarisation, the positive and negative classes correspond to \emph{summary} and \emph{non-summary} sentences, respectively.
These matrix elements can be further combined into informative classification metrics:

\begin{itemize}
    \item Accuracy: Proportion of correctly classified instances out of the total instances.
    Formulated as $\frac{TP + TN}{TP + TN + FP + FN}$.
    \item Precision (or Positive Predictive Value): Proportion of true positive instances out of all instances predicted as positive.
    Formulated as $\frac{TP}{TP + FP}$.
    \item Recall (or Sensitivity): Proportion of true positive instances out of all actual positive instances.
    Formulated as $\frac{TP}{TP + FN}$.
    \item F1-score: Harmonic mean of precision and recall (i.e., the trade-off between the two).
    Formulated as $2 \cdot \frac{Precision \cdot Recall}{Precision + Recall}$.
\end{itemize}



\begin{table}[ht]
    \centering
    \begin{tabular}{c c c}
        \toprule
        \multirow{2}{*}{} & \multicolumn{2}{c}{\textbf{Actual}} \\
        \cmidrule(lr){2-3}
        & \textbf{Positive} & \textbf{Negative} \\
        \midrule
        \textbf{Predicted Positive} & TP & FP \\
        \textbf{Predicted Negative} & FN & TN \\
        \bottomrule
    \end{tabular}
    \caption{Confusion Matrix}
    \label{tab:confusion_matrix}
\end{table}

\begin{figure}[h]
    \centering
    \begin{equation}
        ROUGE-N = \frac{\sum_{S \in R} \sum_{n-gram \in S} count_{match}(n-gram)}{\sum_{S \in R} \sum_{n-gram \in S} count(n-gram)}\label{eq:equation}
    \end{equation}
    \caption{ROUGE-N: N-gram Co-Occurrence Statistics}
    \label{fig:rouge_formula}
\end{figure}
