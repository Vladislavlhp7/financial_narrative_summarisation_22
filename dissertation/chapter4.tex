\section{Evaluation}\label{sec:evaluation}

In general, to assess the quality of a candidate summary $c$, we measure its similarity with the gold summary $c^{*}$
based on their n-gram overlap $R=(c, c^{*})$, where $R$ is the ROUGE-$F_{1}$\footnote{
    \begin{itemize}
        \item We use a slightly different but faster version of ROUGE compared to the official metric~\cite{lin2004rouge}.
              It can be accessed at: \url{https://github.com/pltrdy/rouge}
        \item The official FNS21 evaluation metric is the $F1$-score of ROUGE-2 and we will use it for the final evaluation.
    \end{itemize}
} metric(\cite{lin2004rouge}).
For the FNS21 task due to the extractive nature of our approach we will evaluate our models based on
the ROUGE--maximising $c^{*}_{i}$ gold summary, i.e.,

\begin{figure}[h]
    \centering
    \begin{equation}\label{eq:rouge_max}
        r = \underset{c^{*} \in C^{*}}{\operatorname{argmax}} R(c, c^{*}_{i})
    \end{equation}
    \caption{Candidate summary evaluation as a gold summary ROUGE-maximisation}
    \label{fig:rouge_max}
\end{figure}

The intuition is that by extracting multiple sentences from the report, our generated candidate summary can
retain sentences from \emph{any} of the gold summaries.
Hence, there must be at least one such gold summary where the overlap is maximal.
The practical implications are that two models, $m_{1}$ and $m_{2}$ can produce two different candidate summaries
$c_{1}$ and $c_{2}$, respectively.
Their individual evaluation is based on gold summaries $c^{*}_{1}$ and $c^{*}_{2}$ (which can be the same when the
candidates $c_{1}$ and $c_{2}$ are identical).

Following this evaluation mechanism, we compare our models in terms of their ROUGE metrics in Table~\ref{tab:rouge_performance}.
Please keep in mind that we include official results from the FNS21 competition for comparison, namely the T5-LONG-EXTRACT model
(\cite{orzhenovskii-2021-t5}) and the MUSE model (\cite{litvak-last-2013-multilingual}), which have been
\begin{enumerate*}
    \item trained on more data than our models due to us using the official validation as a test set (see Section~\ref{subsec:data}),
    \item evaluated on the official test set which we did not have access to
\end{enumerate*}.

\begin{table}[ht]
    \centering
    \begin{tabular}{lccc}
        \toprule
        \textbf{Model} & \textbf{ROUGE-1} & \textbf{ROUGE-2} & \textbf{ROUGE-L} \\
        \midrule
            $\star$ FinBERT-base & 0.544 & 0.382 & 0.524 \\
            T5-LONG-EXTRACT (\cite{orzhenovskii-2021-t5})* & 0.54 & 0.38 & 0.52 \\
            FinBERT-base + back-translation & 0.490 & 0.321 & 0.468 \\
            MUSE (\cite{litvak-last-2013-multilingual})* & 0.5 & 0.28 & 0.45 \\
            $\star$ GRU-base + attention + back-translation & 0.276 & 0.106 & 0.249 \\
            GRU-base + back-translation & 0.266 & 0.100 & 0.247 \\
            LexRank & 0.250 & 0.086 & 0.227 \\
            TaxtRank & 0.220 & 0.064 & 0.196 \\
            GRU-base & 0.220 & 0.063 & 0.201 \\
            GRU-base + attention & 0.221 & 0.062 & 0.204 \\
        \bottomrule
    \end{tabular}\caption{ROUGE: Model performance on the FNS21 test dataset.}
    \label{tab:rouge_performance}
\end{table}

We can provide the following commentary on the results:
\begin{itemize}
    \item The best performing model is FinBERT-base (\cite{yang2020finbert}), which is a pre-trained on financial communication documents,
        achieving a ROUGE-2 score of $0.382$, which is \emph{at least as good as the official best-performing model} in the FNS21 competition:
        the T5-LONG-EXTRACT (\cite{orzhenovskii-2021-t5}, Section~\ref{subsec:text-summarisation}).
        Surprisingly, data augmentation does not improve the performance of FinBERT-base, and we believe this was caused by
        the \hyperlink{data_augment_hypothesis}{back-translation hypothesis} we made in Section~\ref{subsec:hyperparameters}.
        Nevertheless, both models outperform the official top-line (\cite{litvak-last-2013-multilingual}) by a significant margin for ROUGE-2.
    \item The GRU-base + attention + back-translation model is the best performing model out of all recurrent neural architectures.
    While preliminary binary classification results did not show any considerable differences between the models, clearly
    \begin{enumerate*}
        \item the attention mechanism helps the model to better recognise the summarising sentences (i.e., attends to the most descriptive linguistic features), and
        \item the back-translation data augmentation significantly improves the practical performance of the model (i.e., the probability distribution of the summarising sentences).
    \end{enumerate*}
    \item At the same time, we acknowledge that the universal summarisation baselines: LexRank and TextRank, outperform
        our simple GRU models, and we attribute this to both:
    \begin{enumerate*}
        \item the lack of sufficient descriptive training data from the positive class (i.e., the summarising sentences, Table~\ref{tab:random_under_sampling}), and
        \item the 90\% random under-sampling of the majority class data (see Section~\ref{subsec:data}).
    \end{enumerate*}
\end{itemize}

